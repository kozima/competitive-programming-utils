\documentclass{scrartcl}
\usepackage{url}
\usepackage{amsmath,amsthm}
\usepackage[all]{xy}

\theoremstyle{definition}
\newtheorem{lemma}{Lemma}
\newtheorem{example}[lemma]{Example}

\begin{document}

\section{木の直径について}

木の直径は任意の点からDFSを二回することで求められることが知られている.これの一般化を考える.

$T$を木とする.以下パスの向きは区別しないことにする.
$P$を$T$のパス全体の集合,$X$を全順序集合とし,$f: P \to X$が与えられているとする.
このとき,$\max_{\pi \in P} f(\pi)$を求める問題を考える.

何の条件もないと無理があるので$f$が以下の条件を満たすことを仮定してみる.
\begin{itemize}
\item $f(\pi \cdot \pi') \ge \max(f(\pi), f(\pi'))$
\item $f(\pi) \le f(\pi') \implies f(\pi \cdot \pi'') \le f(\pi' \cdot \pi'')
  \text{ and } f(\pi'' \cdot \pi) \le f(\pi'' \cdot \pi')$
\end{itemize}
ただし$\pi, \pi' \in P$に対して$\pi \cdot \pi'$はパスの連結であり,
$\pi$と$\pi'$が一つの端点のみを共有するときに限り定義される.

\begin{example}
  $f(\pi)$を$\pi$に含まれる辺の本数とすると,これは直径を求める問題である.
\end{example}

\begin{example}
  各頂点$i$に重み$w_i$が与えられているとする.$f(\pi)$を$\pi$に現れる頂点の重み和とすれば,パスの重み和の最大値を求める問題になる.
\end{example}

以下,頂点$u, v$を結ぶパスを単に$uv$と書く.パスの向きは気にしないので$uv=vu$であることに注意.

\begin{lemma}
  頂点$v$を任意にとり,$f(vw)$を最大化する$w$をひとつとる.また,$x, y$は$f$を最大化するあるパスの両端点とする.このとき$f(xy) = f(wx) = f(wy)$が成り立つ.
\end{lemma}
\begin{proof}
  $xy$-pathに含まれる頂点$u$で$f(vu)$を最小にするものをとる.
  このとき$ux$と$uy$は$u$以外の頂点を共有せず,下図のような状況になる.
  \begin{displaymath}
    \xymatrix{
      w \ar@{-}[r]& v \\
      x \ar@{-}[r]& u \ar@{-}[u] \ar@{-}[r] & y
    }
  \end{displaymath}
  $w$のとり方から
  \begin{displaymath}
    f(wu) = f(wv \cdot vu) \ge f(wv) \ge f(vx)
  \end{displaymath}
  である.よって
  \begin{displaymath}
    f(wx) = f(wu \cdot ux) \ge f(yu \cdot ux) = f(yx)
  \end{displaymath}
  がいえる.逆向きの不等式は$xy$の選び方から明らかに成り立つから$f(wx) = f(yx)$である.
  $f(wy) = f(yx)$も同様に成り立つ.
\end{proof}

$v$を与えたとき,$\max_u f(vu)$を与える$u$はDFSにより求めることができる\footnote{$f$の性質が悪いと$O(|T|)$では求まらないかもしれない.}.上の事実より,こうして求めた$u$は$f$を最大にするパスの端点になるので,$u$を始点としてもう一度DFSをすれば$f$の最大値を求めることができる.

\subsection*{備考}

ARC097-F Monochrome Cat \url{https://atcoder.jp/contests/arc097/tasks/arc097_d}

\end{document}
